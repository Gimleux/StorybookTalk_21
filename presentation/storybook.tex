%%%%%%%%%%%%%%%%%%%%%%%%%%%%%%%%%%%%%%%%%
% Focus Beamer Presentation
% LaTeX Template
% Version 1.0 (8/8/18)
%
% This template has been downloaded from:
% http://www.LaTeXTemplates.com
%
% Original author:
% Pasquale Africa (https://github.com/elauksap/focus-beamertheme) with modifications by 
% Vel (vel@LaTeXTemplates.com)
%
% Template license:
% GNU GPL v3.0 License
%
% Important note:
% The bibliography/references need to be compiled with bibtex.
%
%%%%%%%%%%%%%%%%%%%%%%%%%%%%%%%%%%%%%%%%%

%----------------------------------------------------------------------------------------
%	PACKAGES AND OTHER DOCUMENT CONFIGURATIONS
%----------------------------------------------------------------------------------------

\documentclass{beamer}

\usetheme{focus} % Use the Focus theme supplied with the template
% Add option [numbering=none] to disable the footer progress bar
% Add option [numbering=fullbar] to show the footer progress bar as always full with a slide count

% Uncomment to enable the ice-blue theme
%\definecolor{main}{RGB}{92, 138, 168}
%\definecolor{background}{RGB}{240, 247, 255}
\definecolor{main}{RGB}{255, 71, 133}
\definecolor{background}{RGB}{0, 13, 26}

\input{lstlisting-JavaScript.tex} %Eigene lstlisting Erweiterung

%------------------------------------------------

\usepackage[UTF8]{inputenc}
\usepackage{booktabs} % Required for better table rules
%----------------------------------------------------------------------------------------
%	 TITLE SLIDE
%----------------------------------------------------------------------------------------

\title{Storybook}

\subtitle{UI component explorer for frontend developers}

\author{Gimleux}

\titlegraphic{\includegraphics[scale=.3]{Images/storybook_logo.png}} % Optional title page image, comment this line to remove it

\institute{Software Engineering}

\date{June 01, 2021}

%------------------------------------------------

\begin{document}

%------------------------------------------------

\begin{frame}
	\maketitle % Automatically created using the information in the commands above
\end{frame}

\begin{frame}{Inhalt}
	\begin{itemize}
		\item Was ist Storybook?
		\item Was sind die Vorteile?
		\item Logik in Storybook
		\item (Aufbau der Web-Oberfläche)
		\item Projektinitialisierung
		\item Live-Coding
	\end{itemize}
\end{frame}

%----------------------------------------------------------------------------------------
%	 Was ist Storybook
%----------------------------------------------------------------------------------------

\section{Was ist Storybook?} % Section title slide, unnumbered

\begin{frame}{Was ist Storybook?}
	\begin{columns}
		\column{0.7\textwidth}
		\begin{itemize}
			\item Hilfsmittel bei frontendseitiger Web-Entwicklung
			\item Isolierte Entwicklung von Web-Komponenten
			\begin{itemize}
				\item \emph{component-driven-development}
			\end{itemize}
		\end{itemize}
		
		\column{0.3\textwidth}
		\includegraphics[width=\linewidth]{Images/storybook_logo.png}
	\end{columns}
\end{frame}

\begin{frame}{Was ist eine Story?}
	\begin{itemize}
		\item Beschreibt markanten/interessanten Zustand einer Web-Komponente
			\begin{itemize}
				\item Bei bestimmten Ereignissen
				\item Mit bestimmtem Design
				\item ...
			\end{itemize}
		\item Für eine Web-Komponente sind beliebig viele Stories implementierbar
	\end{itemize}
\end{frame}

%----------------------------------------------------------------------------------------
%	 Vorteile
%----------------------------------------------------------------------------------------

\section{Was sind die Vorteile?} % Section title slide, unnumbered

\begin{frame}{Was sind die Vorteile?}
	\begin{itemize}
		\item Erlaubt an einer einzelnen Komponente zu arbeiten, ohne gesamte Webseite berühren und berücksichtigen zu müssen
		\begin{itemize}
			\item Einfachere \& schnellere Entwicklung der Komponente
			\item Geringere Fehleranfälligkeit
		\end{itemize}
		\item Nutzung von Testdaten innerhalb der Komponente, ohne Einbindung des restlichen Systems
		\item Automatische Dokumentation
	\end{itemize}
\end{frame}

\begin{frame}{Was sind die Vorteile?}
	\begin{itemize}
		\item Isolierte Darstellung von Web-Komponenten
		\item Veränderbarkeit der Komponente über Web-Oberfläche
		\begin{itemize}
			\item Isolierte Präsentation für Product Owner
			\item Simple Live-Anpassungen mit oder durch Product Owner
			\begin{itemize}
				\item => Vereinfachte Zusammenarbeit (intern \& extern)
			\end{itemize}
		\end{itemize}
	\end{itemize}
\end{frame}

\begin{frame}{Was sind die Vorteile?}
	\begin{itemize}
		\item Einzelne Komponenten lassen sich per iFrame einbetten
		\includegraphics[scale=.5]{Images/interactive_demo.png}
	\end{itemize}
\end{frame}

%----------------------------------------------------------------------------------------
%	 Logik hinter Storybook
%----------------------------------------------------------------------------------------
\section{Logik in Storybook} % Section title slide, unnumbered

\begin{frame}{".storybook"-Ordner}
	\begin{columns}	
		\column{0.5\textwidth}
		\includegraphics{Images/file/storybook-project-folder.pdf}
		
		\column{0.5\textwidth}
		".storybook"-Ordner
		\begin{itemize}
			\item Wird bei Installation angelegt
			\item Enthält Grund- und ggf. Add-On-Einstellungen
		\end{itemize}
	\end{columns}
\end{frame}

\begin{frame}{Dateistruktur einer Komponente}
	\begin{columns}	
		\column{0.5\textwidth}
		\includegraphics{Images/file/storybook-component-folder.pdf}
		
		\column{0.5\textwidth}
		\begin{itemize}
			\item Storybook-Dateien enthalten den String "\alert{.stories.}"
			\item Best Practice: <Komponentenname> \alert{.stories.}{}<Typ>
		\end{itemize}
	\end{columns}
\end{frame}

\begin{frame}[fragile]{Aufbau einer Storybook-Datei}
	\begin{itemize}
		\item Import von React sowie der eigentlichen Komponente
	\end{itemize}
	\begin{lstlisting}[language=JavaScript]
		import React from 'react';
		import {Button} from './Button';
	\end{lstlisting}		
\end{frame}

\begin{frame}[fragile]{Aufbau einer Storybook-Datei}
	\begin{itemize}
		\item Storybook-Objekt erstellen
	\end{itemize}
	\begin{lstlisting}[language=JavaScript]
		export default {
			title: 'Atoms/Button',
			component: Button,
			argTypes: {
				color: {control: "color"},
				backgroundColor: {control: 'color'},
			},
		};
	\end{lstlisting}
\end{frame}

\begin{frame}[fragile]{Aufbau einer Storybook-Datei}
	\begin{itemize}
		\item Erstellung eines Templates für die Komponente
		\begin{itemize}
			\item Template kann beliebig über die eigentliche Komponente hinaus erweitert werden
		\end{itemize}
	\end{itemize}
	\begin{lstlisting}[language=JavaScript]
		const Template = (args) => <Button {...args} />;
	\end{lstlisting}
	\begin{lstlisting}[language=JavaScript]
		const CenteredTemplate = (args) => (
		<div className="flex-centered">
		<Button {...args} />
		</div>
		);
	\end{lstlisting}
\end{frame}

\begin{frame}[fragile]{Aufbau einer Storybook-Datei}
	\begin{itemize}
		\item Erstellung von Stories
	\end{itemize}
	\begin{lstlisting}[language=JavaScript]
		export const Default = Template.bind({});
	\end{lstlisting}
	\begin{lstlisting}[language=JavaScript]
		export const NotPrimary = Template.bind({});
		NotPrimary.args = {
			isPrimary: false
		};
	\end{lstlisting}
	\begin{lstlisting}[language=JavaScript]
		export const Large = Template.bind({});
		Large.args = {
			size: 'large'
		};
	\end{lstlisting}
\end{frame}

\begin{frame}[fragile]{Parameter-Typisierung}
	\begin{itemize}
		\item Angabe der erwarteten Argumente und eventueller default-Werte
	\end{itemize}
	\begin{lstlisting}[language=JavaScript]
		Button.propTypes = {
			/**
			* Is this the principal call to action on the page?
			*/
			isPrimary: PropTypes.bool,
			/**
			* How large should the button be?
			*/
			size: PropTypes.oneOf(['small', 'medium', 'large']),
		};
	
		Button.defaultProps = {
			isPrimary: false,
			size: 'medium',
		};
	\end{lstlisting}
\end{frame}

%----------------------------------------------------------------------------------------
%	 Hinzufügen von Tests
%----------------------------------------------------------------------------------------
%\section{Snapshot-Tests} % Section title slide, unnumbered
%\begin{frame}{Snapshot-Tests}
%	\lstset{extendedchars=false,
%		escapeinside=”}
%	\begin{itemize}
%		\item Installation des entsprechenden Add-Ons
%	\end{itemize}
%	\begin{lstlisting}{}[language=JavaScript]
%		npm i -D @storybook/addon-storyshots
%	\end{lstlisting}
%	\begin{itemize}
%		\item Testcode in Projekt einfügen
%	\end{itemize}
%	\begin{lstlisting}{}[language=JavaScript]
%		// storybook.test.js
%		import initStoryshots from '@storybook/addon-storyshots';
%		initStoryshots();
%	\end{lstlisting}
%	\begin{itemize}
%		\item Tests ausführen
%	\end{itemize}
%	\begin{lstlisting}[language=JavaScript]
%		npm test
%	\end{lstlisting}
%\end{frame}

%----------------------------------------------------------------------------------------
%	 Web-Oberflaeche
%----------------------------------------------------------------------------------------
\section{Aufbau der Web-Oberfläche} % Section title slide, unnumbered

\begin{frame}{Aufbau der Web-Oberfläche}
	\includegraphics[width=\textwidth]{Images/web-ui/whole-ui.png}
\end{frame}

\begin{frame}{Seitenleiste}
	\begin{columns}
		\column{0.5\textwidth}
		\includegraphics[height=.8\textheight]{Images/web-ui/sidebar.png}
		
		\column{0.5\textwidth}
		Unterteilung in
		\begin{itemize}
			\item Bereiche (z.B. Atoms)
			\item Komponenten (z.B. Button)
			\item Stories (z.B. Large)
		\end{itemize}
	\end{columns}
\end{frame}

\begin{frame}{Canvas}
	\begin{columns}
		\column{0.5\textwidth}
		\begin{itemize}
			\item Ansicht der Komponente
			\item Kontrollbereich, über welchen die Komponente direkt verändert werden kann
		\end{itemize}
		
		\column{0.5\textwidth}
		\includegraphics[width=\linewidth]{Images/web-ui/canvas.png}
	\end{columns}
\end{frame}

\begin{frame}{Docs}
	\begin{columns}
		\column{0.5\textwidth}
		\begin{itemize}
			\item Übersicht der Komponente, ihrer Stories und Parameter
			\item Einsicht des zur Implementierung benötigten Codes
			\item Details zu Parametern der Komponente
		\end{itemize}
		
		\column{0.5\textwidth}
		\includegraphics[width=\linewidth]{Images/web-ui/docs.png}
	\end{columns}
\end{frame}

%----------------------------------------------------------------------------------------
%	 Aufsetzen
%----------------------------------------------------------------------------------------
\section{Projekt-Initialisierung} % Section title slide, unnumbered

\begin{frame}{Projekt initialisieren}
	\begin{itemize}
		\item npx create-react-app [--use-npm]
		\item npx sb init
	\end{itemize}
\end{frame}

\begin{frame}{Storybook ausführen}
	\begin{itemize}
		\item cd create-react-app
		\item npm run storybook
	\end{itemize}
\end{frame}

%----------------------------------------------------------------------------------------
%	 Live-Coding
%----------------------------------------------------------------------------------------
\section{Live-Coding} % Section title slide, unnumbered

%------------------------------------------------

%----------------------------------------------------------------------------------------
%	 CLOSING/SUPPLEMENTARY SLIDES
%----------------------------------------------------------------------------------------

\appendix

%------------------------------------------------

%\begin{frame}{Backup Slide}
%	This is a backup slide, useful to include additional materials to answer questions from the audience.
%	\vfill
%	The package \texttt{appendixnumberbeamer} is used to refrain from numbering appendix slides.
%\end{frame}

%----------------------------------------------------------------------------------------

\end{document}
